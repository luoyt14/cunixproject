%\documentclass[a4paper]{article}
%\usepackage{beamerarticle}
\documentclass[10pt]{beamer}

\usepackage{graphicx}
\usepackage{float} 
\usepackage{subfigure}
\setbeamertemplate{caption}[numbered]
\usepackage{multirow}
\usepackage{indentfirst}

\makeatletter
\renewcommand \theequation {%
	\ifnum \c@section>\z@ \@arabic\c@section.\fi \ifnum \c@subsection>\z@
	\@arabic\c@subsection.\fi\ifnum \c@subsubsection>\z@
	\@arabic\c@subsubsection.\fi\@arabic\c@equation}
\@addtoreset{equation}{section}
\@addtoreset{equation}{subsection}
\makeatother
\setcounter{section}{-1}

\renewcommand\thefigure{\thesection.\arabic{figure}}
\makeatletter
\@addtoreset{figure}{section}
\makeatother

\allowdisplaybreaks

%\usepackage[dvipsnames,table,xcdraw]{xcolor}
%\usepackage{beamerarticle}
\usetheme{Berlin}
%\usetheme{Boadilla}
%\usetheme{Hannover}
\useoutertheme[height=1.5cm]{sidebar}
\usecolortheme{spruce}
\usecolortheme{lily}
\usefonttheme{professionalfonts}

%\usepackage{ctex}
\usepackage{xeCJK}
\usepackage{caption}
\setCJKmainfont{KaiTi}
\setmainfont{Times New Roman}
\usepackage{amsmath}
\usepackage{algorithm}
\usepackage{algorithmicx}
\usepackage{algpseudocode}
\usepackage{listings}
%\floatname{algorithm}{算法}
\renewcommand{\algorithmicrequire}{\textbf{输入:}}  
\renewcommand{\algorithmicensure}{\textbf{输出:}} 

\lstset{
	columns=fixed,       
%	numbers=left,                                        % 在左侧显示行号
	numberstyle=\tiny\color{gray},                       % 设定行号格式
	frame=none,                                          % 不显示背景边框
	backgroundcolor=\color[RGB]{245,245,244},            % 设定背景颜色
	keywordstyle=\color[RGB]{40,40,255},                 % 设定关键字颜色
	numberstyle=\footnotesize\color{darkgray},           
	commentstyle=\it\color[RGB]{0,96,96},                % 设置代码注释的格式
	stringstyle=\rmfamily\slshape\color[RGB]{128,0,0},   % 设置字符串格式
	showstringspaces=false,                              % 不显示字符串中的空格
	language=c++,                                        % 设置语言
}

\AtBeginSection[]{
	\begin{frame}{OUTLINE}
	\tableofcontents[currentsection]
	\end{frame}
}

\AtBeginSubsection[]{
	\begin{frame}{OUTLINE}
	\tableofcontents[currentsection,currentsubsection]
    \end{frame}
}
\setlength{\parindent}{2em}
\title{Yet Another Unix Shell }
\author{\href{mailto:luoyt14thu@gmail.com}{罗雁天 2018310742}}
\date{\today}
\logo{\includegraphics[height=1.5cm]{Tsinghua2.png}}
%\hyperlinkdocumentend{\logo{\includegraphics[height=1.5cm]{Tsinghua2.png}}}

\begin{document}
\begin{frame}{C/Unix程序设计大作业}
\titlepage
\end{frame}

\begin{frame}{目录}
\tableofcontents
\end{frame}

\section{简介}

\begin{frame}{简介}
本次大作业实现了一个命令行解释器yaush,能够实现如下功能:

\begin{itemize}
	\item 用户输入命令与参数,能够正常执行命令;
	\item 输入、输出重定向到文件;
	\item 管道;
	\item 后台执行程序;
	\item 作业控制(jobs,bg,fg);
	\item 历史命令(history)(TODO);
	\item 文件名tab补全,各种快捷键(TODO);
	\item 环境变量、简单脚本;
\end{itemize}
\end{frame}

\section{实现细节}
\subsection{命令结构设计}
\begin{frame}[fragile]
\frametitle{命令结构设计}
在实验中,设置了一个结构体来保存从输入解析到的命令结构:
\begin{lstlisting}
struct cmd {
    struct cmd* next; //下一个命令
    int begin, end; //命令的开始位置和结束位置
    int argc; //命令和参数的总个数
    char lredir, rredir; //输入、输出重定向的标识
    char toFile[MAX_PATH_LENGTH]; //输出文件
    char fromFile[MAX_PATH_LENGTH]; //输入文件
    char *args[MAX_ARG_NUM]; //命令的参数
    char bgExec; //是否后台执行
};
\end{lstlisting}
\end{frame}

\subsection{输入命令处理}
\begin{frame}{输入命令处理}

\end{frame}

\section{实验结果}
\subsection{执行步骤}
\begin{frame}
\frametitle{执行步骤}
进入code/文件夹下,输入"make"进行编译,然后输入"./main"执行进入yaush模式,如图\ref{init}所示:
\begin{figure}[htbp]
	\centering
	\includegraphics[width=0.7\textwidth]{images/init}
	\caption{\label{init}编译运行文件执行结果}
\end{figure}
\end{frame}

\subsection{命令执行}
\begin{frame}{正确执行简单命令}
在此我们演示几个较为简单地命令(ls, sl, cd等),如图所示:
\end{frame}

\begin{frame}{输入输出重定向到文件}
在此我们演示将命令输入重定向和输出重定向的功能,如图所示:
\end{frame}

\begin{frame}{管道操作}
在此我们通过管道操作符'|'演示管道操作,如图所示:
\end{frame}

\begin{frame}{后台执行程序}
在此我们对比演示'sleep 10'和'sleep 10 \&'两个命令的区别,如图所示:
\end{frame}

\begin{frame}{环境变量设置}
在此我们执行export设置环境变量并且用echo输出环境变量作为演示,如图所示:
\end{frame}
\begin{frame}{简单脚本执行}
在此我们首先使用vim建立一个脚本文件,然后在yaush中执行此脚本文件,如图所示:
\end{frame}

\section{总结}
\begin{frame}{总结}
本次大作业简单地实现了一个命令执行程序(shell),但是在使用自己shell的过程中发现和自带的bash、zsh等成熟的shell相比较还是差很多。通过本次实验,充分复习了上课学到的知识也通过自己的查阅资料学习到了很多新的东西,希望在以后能够用到自己的科研以及工作之中。但是由于考试的原因,实现的shell并不完美,希望之后有时间能够在以下几个方面进行优化。

\begin{itemize}
	\item 能够正确使用上下左右箭头进行调整;
	\item 能够支持中文;
	\item 能够使用history命令以及实现自动补全、快捷键等。
\end{itemize}
\end{frame}

\begin{frame}{致谢}
\begin{itemize}
	\item 感谢老师一学期以来的辛勤讲述;
	\item 感谢助教一学期认真批改作业以及对作业的指导;
\end{itemize}
\end{frame}



\end{document}